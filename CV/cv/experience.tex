%-------------------------------------------------------------------------------
%	SECTION TITLE
%-------------------------------------------------------------------------------
\cvsection[atom]{科研 \& 项目}


%-------------------------------------------------------------------------------
%	CONTENT
%-------------------------------------------------------------------------------
\begin{cventries}

%---------------------------------------------------------
  \cventry
    {深度学习~~医学图像处理~~图像分割~~\href{https://vtk.org/}{\textbf{PyTorch}}} % keywords
    {基于卷积编解码网络(CED)的膝关节骨结构分割} % project name
    {复旦大学} % Location
    {2019.5 - 2019.12} % Date(s)
    {
      \begin{cvitems} % Description(s) of tasks/responsibilities
        \item {探究使用深度学习的膝关节关键骨结构分割方法及应用}
        \item {基于U型CED网络结构,引入全局信息支路,解决基于块模型的全局信息缺失问题}
        \item {基于\href{https://vtk.org/}{PyTorch}实现训练及推理PipeLine}\\
        \textbf{结果}~~同一公开数据集上取得已知文献中最高分割精度【Dice系数】骨0.96 软骨0.75,且能有效分割临床实际数据
      \end{cvitems}
    }

%---------------------------------------------------------
  \cventry
    {点云配准~~跨模态配准~~\href{https://itk.org/}{\textbf{ITK}}~~\href{https://pointclouds.org/}{\textbf{PCL}}} % Job title
    {基于分割结构对齐的膝关节MR-CT跨模态配准} % keywords
    {复旦大学} % project name
    {2019.11 - 2020.4} % Date(s)
    {
      \begin{cvitems} % Description(s) of tasks/responsibilities
        \item {探究膝关节医学图像的MR-CT跨模态配准方法}
        \item {基于MR和CT的分割标签采样对应表面点云,计算并匹配点云的特征点,计算变换矩阵完成配准}
        \item {基于C++ \& \href{https://itk.org/}{ITK} \& \href{https://pointclouds.org/}{PCL}实现点云提取、特征计算及配准算法}\\
        \textbf{结果}~~配准精度及鲁棒性高于基于体素灰度相似度的方法及基于ICP的配准方法
      \end{cvitems}
    }

%---------------------------------------------------------
  \cventry
    {医学图像可视化~~\href{https://vtk.org/}{\textbf{VTK}}~~\href{https://www.qt.io/cn}{\textbf{QT}}} % Job title
    {\href{http://ilab.fudan.edu.cn/med7}{人脑解剖与影像结构虚拟仿真实验教学系统}-数字脑2D\&3D可视化部分} % 
    {复旦大学} % Organization
    {2019.2 - 2019.5} % Date(s)
    {
      \begin{cvitems} % Description(s) of tasks/responsibilities
        \item {探究医学图像的三维及二维可视化方法}
        \item {借鉴开源医学图像计算和可视化软件平台 \href{https://www.slicer.org/}{3D Slicer},使用\href{https://www.qt.io/cn}{QT} \& \href{https://vtk.org/}{VTK}框架,实现脑图像的断层及三维可视化}\\
        \textbf{结果}~~实现可视化软件,能对脑结构进行三维表面及二维切面展示并显示结构标签,并能进行基础交互及三维二维联合交互\href{https://github.com/Littlehhh/Medical-Image-Visualization}{\textbf{【软件截图】}}
      \end{cvitems}
    }

%---------------------------------------------------------
  \cventry
    {灰度共生矩阵~~模糊聚类~~ \href{https://opencv.org/}{\textbf{OpenCV}}} % Job title
    {基于X波段雷达遥感数据的海面溢油反演算法及OpenCV实现} % Organization
    {中国海洋大学 \& 国家海洋局第一海洋研究所} % Location
    {2017.10 - 2018.4} % Date(s)
    {
      \begin{cvitems} % Description(s) of tasks/responsibilities
        \item {探究根据船载x波段雷达数据有效反演海表面溢油的方法}
        \item {将雷达数据转换为数字图像,采用灰度共生矩阵提取图像的纹理特征,进行模糊聚类输出预警结果}
        \item {使用C++\&OpenCV实现溢油反演算法,在实际船载溢油实验中应用}\\
        \textbf{结果}~~能对海表面出现溢油进行有效预警,并且能计算溢油区域面积
      \end{cvitems}
    }

%---------------------------------------------------------
\end{cventries}
